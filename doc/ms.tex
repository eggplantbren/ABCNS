\documentclass[a4paper, 11pt]{article}
\usepackage{graphicx}
\usepackage{natbib}
\usepackage{amsmath}
\usepackage{dsfont}
\usepackage[left=3cm,top=3cm,right=3cm]{geometry}

\newcommand{\params}{\boldsymbol{\theta}}	% The unknown parameters
\newcommand{\data}{\boldsymbol{D}}  % The data
\newcommand{\dx}{d^N\mathbf{\params}} % Volume element in parameter space
\renewcommand{\topfraction}{0.85}
\renewcommand{\textfraction}{0.1}
\parindent=0cm

\title{Approximate Bayesian Computation with Nested Sampling}
\author{Brendon J. Brewer}

\begin{document}
\maketitle

\section{Bayesian inference}
By the product rule of probability theory, the prior $p(\params)$ and the
conditional prior for the data $p(\data | \params)$ imply a joint prior
for $\params$ and $\data$:
\begin{equation}
\begin{array}{lr}
p(\params, \data) = p(\params)p(\data|\params)
\end{array}
\end{equation}


\begin{table}[ht!]
\centering
\small
\begin{tabular}{lll}
\hline
Bayesian computation		&		Statistical mechanics		&		ABC\\
\hline
Parameter space	$\Theta$	&		Phase space	$\Omega$ or configuration space $\Gamma$ 			& (Parameter$\times$Data) space $\Theta \times \mathcal{D}$\\		
Marginal likelihood / evidence	&	Partition function at $T=1$	&\\


\end{tabular}
\caption{\it The relationship between standard Bayesian computation, statistical
mechanics, and ABC. In each case Monte Carlo methods are used to calculate
integrals over a space. In standard Bayesian computation, it is the parameter
space of a model that is usually of interest, whereas in ABC it is the space
of possible parameter values {\it and} data sets.
\label{tab:relation}}
\end{table}

\end{document}

